\documentclass{vldb}
\usepackage{graphicx}
\usepackage{balance} 
\usepackage{url}

\begin{document}

\title{Providing better confidentiality and authentication on the Internet using Namecoin and MinimaLT}

\numberofauthors{1}
\author{
\alignauthor
Frederic Jacobs\\
\affaddr{www.fredericjacobs.com}\\
\email{me@fredericjacobs.com}
}

\maketitle

\begin{abstract}
In this paper, we introduce a duo of improvements for the Internet that would lead to better security. The authentication model on the Internet is broken and TLS connections have a considerable overhead. We try to address those issues with changes at both the application layer, introducing a replacement for the DNS system, and at the transport layer, a drop-in replacement for TCP built on top of UDP that requires no changes at the network layer.
\end{abstract}

\section{Introduction}

\subsection{Defining user privacy}
The solutions brought forward in this paper are attempts to fix confidentiality and authentication on the Internet. Anonymity is not provided. An attacker could still get a significant amount of metadata. Unfortunately, because MinimaLT runs over UDP you can't connect through the Tor network.\footnote{If MinimaLT proves to be a safer and faster alternative to TLS, I imagine that the Tor project would look into implementing it to speed up the network and make relay connections safer.}
\subsection{Motivation}

When the Internet was designed at DARPA, the primary goal was to design a system that could provide interconnection between multiple computers. The Web then came by with the motivation to be able to freely exchange information. The Internet has mainly been used for open communications. Any computer on the network could request files. But, over time, people started trusting the internet more and more and with the appearance of services. But the Internet grew so quickly out of what DARPA proposed for a trusted environment. The Internet was never designed to be ran by so many different entities and the threat model implied that you had to trust the entities running the infrastructure of the internet.

\section{Domain names and authenticity}

In today's model, if I want to load a page from \emph{facebook.com}, my computer will have to first get the Domain Name System records matching that domain. DNS was designed in a hierarchical way and TLD registrations are handled by a single organisation, the ICANN.

So what is wrong with DNS?

When the original Domain Name System was designed, it did not include security; instead it was designed to be a scalable distributed system. The DNSSEC attempted to add security, while maintaining backwards compatibility. Those security extensions added public key signing to DNS zones but who is signing those zones? The DNS Root Zone, of course. You will then have to trust those too. We thus consider DNSSec as an attempt to try to fix a broken system. We want to design a distributed system where anyone can register a domain but without having a central registration authority.

But how can we verify authenticity? 
Even if my domain name system returns the right IP address how do I know for sure that I'm establishing a connection with the client I want. Today, we are using another hierarchical system to verify authenticity, namely SSL certificates. This means that in addition to trusting ICANN, we will have to trust hundreds of Root Certificate Authorities that are shipped with out browsers.\cite{mozillaSSL}

If only one of those 100s of CA gets compromised, it could result in the man-in-the-middling of any user without any warning since a root certification authority can generate a fake valid certificate for any website. This is what happened in Iran after the Dutch certificate authority DigiNotar got compromised\cite{diginotarHack}.

Now that we are convinced that the hierarchical trust model of the internet is broken, what measures have already been taken to fix authentication on the Internet?

\subsubsection{Certificate Pinning}
Works great but not scalable - first fix.
\subsubsection{DANE}
Works great but still having a centralised registry / domain registrars
DNSSec
\subsubsection{Tor Hidden Services}
Good, but as seen in Zooko's triangle we don't have the unique address.
\subsubsection{Squaring Zooko's triangle}
Bitcoin chain

Zooko's triangle


\subsection{Known Issues with this new model}

\section{Transport security}
What's wrong with tcp is slow and insecure. How to move away from it? Well, we don't really have any other option to base it on UDP.
But we love reliability!

Minimalt 

Doesn't solve anonimity ==> Tor

Issues with too big frames for firewalls?
\section{The {\secit Body} of The Paper}


\section{Conclusions}

% ensure same length columns on last page (might need two sub-sequent latex runs)
//\balance

%ACKNOWLEDGMENTS are optional
\section{Acknowledgments}
Most of the Namecoin research is based on Greg Slepak's work.
\bibliographystyle{abbrv}
\bibliography{bibliography}
\subsection{References}

\begin{appendix}
Define Perfect forward secrecy

Using Elliptic curve crypto but not backdoored.

Impact on censorship( when encrypted blobs can be transferred)
\section{Final Thoughts}

\end{appendix}

\end{document}
